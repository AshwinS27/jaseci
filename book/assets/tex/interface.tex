\chapter{Interfacing a Jaseci Machine}
\jacdot{dia_api_server_client}{.5}{Jaseci Interface Architecture}
Now that we know what Jaseci is all about, next lets roll up our sleeves and jump in. One of the best ways to jump into Jaseci world is to gather some sample Jac programs and start tinkering with them.
\par
Before we jump right into it, it's important to have a bit of an understanding of the the way the interface itself is architected from in the implementation of the Jaseci stack. Jaseci has a module that serves as its  the core interface (summarized in Table~\ref{tab:jsAPI}) to the Jaseci machine. This interface is expressed as a set of method functions within a python class in Jaseci  called \texttt{master}. (By the way, don't worry, it's ok to use ``master'', its not racialist, see Rant~\ref{rant:racistmaster} for more context). The `client' expressions of that interface in the forms of a command line tool \texttt{jsctl} and a server-side REST API built using Django~\footnote{Django ~\cite{django} is a Python web framework for rapid development and clean, pragmatic design}. Figure~\ref{dot:dia_api_server_client} illustrates this architecture representing the relationship between core APIs and client side expressions.
\printtabJSAPI
If I may say so myself the code architecture of interface generation from function signatures is elegant, sexy, and takes advantage of the best python has to offer in terms of its support for introspection. With this approach, as the set of functions and their semantics change in the \texttt{master} API class, both the JSCTL Cli tool and the REST Server-side API changes. We dig into this and tons more in the Part~\ref{part:crafting}, so we'll leave the discussion on implementation architecture there for the moment. Lets jump right into how we get started playing with some \gls{leet} Jaseci \gls{haxor}ing. First we start with JSCTL then dive into the REST API.


\section{JSCTL: The Jaseci Command Line Interface}
JSCTL or \texttt{jsctl} is a command line tool that provides full access to Jaseci. This tool is installed alongside the installation of the Jaseci Core package and should be accessible from the command line from anywhere. Let's say you've just checked out the Jaseci repo and you're in head folder. You should be able to execute the following.
\par
\shellout{jsctl_setup.shell}
\par
Here we've installed the Jaseci python package that can be imported into any python project with a directive such as \texttt{import jaseci}, and at the same time, we've installed the \texttt{jsctl} command line tool into our OS environment. At this point we can issue a call to say \texttt{jasctl --help} for any working directory.

\subsection{The Very Basics: CLI vs Shell-mode, and Session Files }
This command line tool provides full access to the Jaseci core APIs via the command line, or a shell mode. In shell mode, all of the same Jaseci API functionally is available within a single session. To invoke shell-mode, simply execute \texttt{jsctl} without any commands and jsctl will enter shell mode as per the example below.
\par
\shellout{jsctl_shell_mode.shell}
\par
Here we launched \texttt{jsctl} directly into shell mode for a single session and we can issue various calls to the Jaseci API for that session. In this example we issue a single call to \texttt{graph create}, which creates a graph within the Jaseci session with a single root node, then exit the shell with \texttt{exit}.
\par
The exact behavior can be achieved without ever entering the shell directly from the command line as shown below.
\par
\shellout{jsctl_cli_mode.shell}
\par
All such calls to Jaseci's API can be issued either through shell-mode and CLI mode.
\subsection{A Simple Workflow for Tinkering}
\section{Jaseci Rest API}