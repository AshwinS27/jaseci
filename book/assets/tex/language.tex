\chapter{The Jac Programming Language}
\section{Getting the Basics Out of the Way}
\subsection{Basic Arithmetic Operations}
The simplest math operations in Jac.


\begin{lstlisting}[caption={Basic arithmetic operations}]
walker init {
    a = 4 + 4;
    b = 4 * -5;
    c = 4 / 4;  # Evaluates to a floating point number
    d = 4 - 6;
    e = a + b + c + d;
    std.out(a, b, c, d, e);
}
    \end{lstlisting}

\begin{lstlisting}[language=shell]
8 -20 1.0 -2 -13.0
        \end{lstlisting}


Additionally, Jac supports power and modulo operations.

\begin{lstlisting}[caption={Additional arithmetic operations}]
walker init {
    a = 4 ^ 4; b = 9 % 5; std.out(a, b);
}
    \end{lstlisting}

\begin{lstlisting}[language=shell]
256 4
        \end{lstlisting}
