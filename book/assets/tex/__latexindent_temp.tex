\chapter{Interfacing a Jaseci Machine}
\jacdot{dia_api_server_client}{.5}{Jaseci Interface Architecture}
Now that we know what Jaseci is all about, next lets roll up our sleeves and jump in. One of the best ways to jump into Jaseci world is to gather some sample Jac programs and start tinkering with them.
\par
Before we jump right into it, it's important to have a bit of an understanding of the the way the interface itself is architected from in the implementation of the Jaseci stack. Jaseci has a module that serves as its  the core interface (summarized in Table~\ref{tab:jsAPI}) to the Jaseci machine. This interface is expressed as a set of method functions within a python class in Jaseci  called \texttt{master}. (By the way, don't worry, it's ok to use ``master'', its not racialist, see Rant~\ref{rant:racistmaster} for more context). The `client' expressions of that interface in the forms of a command line tool \texttt{jsctl} and a server-side REST API built using Django~\footnote{Django ~\cite{django} is a Python web framework for rapid development and clean, pragmatic design}. Figure~\ref{dot:dia_api_server_client} illustrates this architecture representing the relationship between core APIs and client side expressions.
\printtabJSAPI
If I may say so myself the code architecture of interface generation from function signatures is elegant, sexy, and takes advantage of the best python has to offer in terms of its support for introspection. With this approach, as the set of functions and their semantics change in the \texttt{master} API class, both the JSCTL Cli tool and the REST Server-side API changes. We dig into this and tons more in the Part~\ref{part:crafting}, so we'll leave the discussion on implementation architecture there for the moment. Lets jump right into how we get started playing with some \gls{leet} Jaseci \gls{haxor}ing. First we start with JSCTL then dive into the REST API.


\section{JSCTL: The Jaseci Command Line Interface}
JSCTL or \texttt{jsctl} is a command line tool that provides full access to Jaseci. This tool is installed alongside the installation of the Jaseci Core package and should be accessible from the command line from anywhere. Let's say you've just checked out the Jaseci repo and you're in head folder. You should be able to execute the following.
\par
\shellout{jsctl_setup.shell}
\par
Here we've installed the Jaseci python package that can be imported into any python project with a directive such as \texttt{import jaseci}, and at the same time, we've installed the \texttt{jsctl} command line tool into our OS environment. At this point we can issue a call to say \texttt{jasctl --help} for any working directory.

\subsection{The Very Basics: CLI vs Shell-mode, and Session Files }
This command line tool provides full access to the Jaseci core APIs via the command line, or a shell mode. In shell mode, all of the same Jaseci API functionally is available within a single session. To invoke shell-mode, simply execute \texttt{jsctl} without any commands and jsctl will enter shell mode as per the example below.
\par
\shellout{jsctl_shell_mode.shell}
\par
Here we launched \texttt{jsctl} directly into shell mode for a single session and we can issue various calls to the Jaseci API for that session. In this example we issue a single call to \texttt{graph create}, which creates a graph within the Jaseci session with a single root node, then exit the shell with \texttt{exit}.
\par
The exact behavior can be achieved without ever entering the shell directly from the command line as shown below.
\par
\shellout{jsctl_cli_mode.shell}
\par
All such calls to Jaseci's API (summarized in Table~\ref{tab:jsAPI}) can be issued either through shell-mode and CLI mode.
\paragraph{Session Files}
At this point, it's important to undestand how sessions work. In a nutshell, a session captures the complete state of a jaseci machine. This state includes the status of memory, graphs, walkers, configurations, etc. The complete state of a Jaseci machine can be captured in a \texttt{.session} file. Every time state changes for a given session via the \texttt{jsctl} tool the assigned session file is updated. If you've been following along so far, try this.
\par
\shellout{session_default.shell}
\par
Note from the first call to \texttt{ls} we have a session file that has been created call \texttt{js.session}. This is the default session file \texttt{jsctl} creates and utilizes when called either in cli mode or shell mode. After listing session files, notices the call to \texttt{graph list} which lists the root nodes of all graphs created within a Jaseci machine's state. Note \texttt{jsctl} lists two such graph root nodes. Indeed these nodes correspond to the ones we've just created when constrasting cli mode and shell mode above. Having these two graphs demonstrates that across both instantiations of \texttt{jsctl} the same session, \texttt{js.session}, is being used. Now try the following. 
\par
\shellout{new_session.shell}
\par
Here we see that we can use the \texttt{-f} or \texttt{--filename} flag to specify the session file to use. In this case we list the graphs of the session corresponding to \texttt{mynew.session} and see the JSON representation of an empty list of objects. We then list session files and see that one was created for \texttt{mynew.session}. If we were to now type \texttt{jsctl --filename js.session graph list}, we would see a list of the two graph objects that we created earlier. 
\paragraph{In-memory mode}
Its important to note that there is also an in-membory mode that can be created buy using the \texttt{-m} or \texttt{--mem-only} flags. This flag is particularly useful when you'd simply like to tinker around with a machine in shell-mode or you'd like to script some behavior to be executed in Jac and have no need to maintain machine state after completion. We will be using in memory session mode quite a bit, so you'll get a sense of its usage throughout this chapter. Next we actually see a workflow for tinkering. 

\subsection{A Simple Workflow for Tinkering}

As you get to know Jaseci and Jac, you'll want to try things and tinker a bit. In this section, we'll get to know \texttt{jsctl} for this play. A typical flow will involve jumping into shell-mode, writing some code, running that code to observe output and the state of the graph, and rendering that graph in dot to see it's visuallization. Before we jump right in, let me strongly encourage you install Graphviz. Graphviz is open source graph visualization software package that includes a handy dandy command line tool call \texttt{dot}. Dot is also a standardized and open graph description language that is a key primitive of Graphviz and the \texttt{dot} tool in Graphviz takes dot code and renders it nicely. Graphviz is super easy to install. In Ubuntu simply type \texttt{sudo apt install graphviz}, or on mac type \texttt{brew install graphviz} and you're done! You should be able to call \texttt{dot} from the command line. 

\section{Jaseci Rest API}