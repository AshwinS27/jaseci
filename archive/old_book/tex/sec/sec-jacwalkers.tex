Define and describe walker in Jac

Introduce graphs, nodes, edges


\begin{description}
    \begin{lstlisting}[caption={Simple walker creating and connected nodes}]
walker init {
    node1 = spawn node;
    node2 = spawn node;
    node1 <-> node2;
    here --> node1;
    node2 <-- here;
}
    \end{lstlisting}
    \item[Output] \texttt{}
          \begin{lstlisting}[language=shell]
        \end{lstlisting}
    \item[Description] \texttt{}
\end{description}


\begin{description}
    \begin{lstlisting}[caption={Creating named node types}]
node person;
edge assists;
edge family;

walker init {
    node1 = spawn node::person;
    node2 = spawn node::person;
    node1 <-[family]-> node2;
    here -[friend]-> node1;
    node2 <-[friend]- here;

    # named and unnamed edges and nodes can be mixed
    node2 --> here;
}
    \end{lstlisting}
    \item[Output] \texttt{}
          \begin{lstlisting}[language=shell]
        \end{lstlisting}
    \item[Description] \texttt{}
\end{description}


\begin{description}
    \begin{lstlisting}[caption={Connecting nodes within \texttt{spawn} statement}]
node person;
edge assists;
edge family;

walker init {
    node1 = spawn here -[friend]-> node::person;
    node2 = spawn node1 <-[family]-> node::person;
    here -[friend]-> node2;
}
    \end{lstlisting}
    \item[Output] \texttt{}
          \begin{lstlisting}[language=shell]
        \end{lstlisting}
    \item[Description] \texttt{}
\end{description}


\begin{description}
    \begin{lstlisting}[caption={Chaining node connections using the connect operator}]
node person;
edge assists;
edge family;

walker init {
    node1 = spawn node::person;
    node2 = spawn node::person;
    node2 <-[friend]- here -[friend]-> node1
          <-[family]-> node2;
}
    \end{lstlisting}
    \item[Output] \texttt{}
          \begin{lstlisting}[language=shell]
        \end{lstlisting}
    \item[Description] \texttt{}
\end{description}

\begin{description}
    \begin{lstlisting}[caption={Walkers spawning other walkers}]
node person;
edge assists;
edge family;

walker family_ties {
    for i in -[family]->:
        std.out(here, ' is related to ', i);
}

walker init {
    node1 = spawn here -[friend]-> node::person;
    node2 = spawn node1 <-[family]-> node::person;
    here -[friend]-> node2;
    spawn here walker::family_ties;
}
    \end{lstlisting}
    \item[Output] \texttt{}
          \begin{lstlisting}[language=shell]
        \end{lstlisting}
    \item[Description] \texttt{}
\end{description}



\begin{description}
    \begin{lstlisting}[caption={Getting returned values from spawned walkers}]
node person;
edge assists;
edge family;

walker family_ties {
    has anchor fam_nodes;
    fam_nodes = -[family]->:
}

walker init {
    node1 = spawn here -[friend]-> node::person;
    node2 = spawn node1 <-[family]-> node::person;
    here -[friend]-> node2;
    fam = spawn here walker::family_ties;
    for i in fam:
        std.out(here, ' is related to ', i);
}
    \end{lstlisting}
    \item[Output] \texttt{}
          \begin{lstlisting}[language=shell]
        \end{lstlisting}
    \item[Description] \texttt{}
          \begin{remark}
              \begin{tBox}
                  Remember \texttt{spawn} statements are expressions so they can be used as such, e.g.,
                  \begin{lstlisting}
for i in spawn here walker::family_ties:
    std.out(here, ' is related to ', i);
        \end{lstlisting}
              \end{tBox}
          \end{remark}
\end{description}

