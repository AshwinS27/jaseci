\chapter{Walkers Navigating Graphs}



%%%%%%%%%%%%%%%%%%%%%%%%%%%%%%
\section{Taking Edges (and Nodes?)}

\subsection{Basic Walks}

%%%%%%%
\par
\jaccode{jac_take.jac}{Basic example of walker traveling graph
}

%%%%%%%
\par
\jaccode{jac_take_fanout.jac}{Fan out style takes
}

\subsection{Breadth First vs Depth First Walks}
\par
\jacdotnw{jac_walk_bfs}{.7}{Graph in memory for JC~\ref{jac:jac_walk_bfs.jac}}
If you've played with the basic \texttt{take} command a bit you would notice that by default it results in a breadth first traversal of a graph.
However, the \texttt{take} command is indeed quite flexible.
You can specify an orientation of the \texttt{take} command to navigate with a breadth first or a depth first traversal.
\par
\jaccode{jac_walk_bfs.jac}{Breadth first navigation with take vs depth first}
\par
Take for example the program shown in JC~\ref{jac:jac_walk_bfs.jac}.
First we observe the definition of a static three level binary tree with the graph \texttt{example} on line 3.
This is a vanilla structure as depicted in Figure~\ref{dot:jac_walk_bfs}.
Two walkers are present in this example, one walker \texttt{walk\_with\_breadth}, for which we observe a call to \texttt{take:bfs -->;} indicating a breadth first traversal, and another walker \texttt{walk\_with\_breadth}, for which we observe a call to \texttt{take:bfs -->;} indicating a depth first traversal.
\par
As can be seen in its output,
\par
\shellout{jac_walk_bfs.jac.output}
The print statement on line 34 demonstrate the order of nodes visited correspond to the specified traversal order.
\par
Additionally, the short hand of \texttt{take:b -->;}, or \texttt{take:d -->;} could be used to specify breadth first or depth first traversals respectively.

%%%%%%%%%%%%%%%%%%%%%%%%%%%%%%
\section{Ignoring and Deleting}

%%%%%%%
\par
\jaccode{jac_ignore.jac}{Ignoring edges during walk
}

%%%%%%%
\par
\jaccode{jac_destroy.jac}{Destorying nodes/edges during walk
}

%%%%%%%%%%%%%%%%%%%%%%%%%%%%%%
\section{Reporting Back as you Travel}

%%%%%%%
\par
\jaccode{jac_report.jac}{Building reports as you walk
}