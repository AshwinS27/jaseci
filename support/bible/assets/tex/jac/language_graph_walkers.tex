\chapter{Graphs, Architypes, and Walkers in Jac}



%%%%%%%%%%%%%%%%%%%%%%%%%%%%%%
\section{Structure of a Jac Program}

\par [Introduce structure of a jac program] \par
\par [Specify the differnce between graph architypes, graph instantiations, and walkers]\par
\par [Present simple program that utilizes the structures]\par
\par [Present variations on articulating the same program]\par
\par [Code blocks]\par

\begin{nerd}
    Grammar~\ref{gram:jac.g4:3:16} shows the lines from the formal grammar for Jac that presents the high level structure of a Jac program.
    \gramcoderange{jac.g4}{Jac grammar clip relevant to arithmetic}{3}{16}
\end{nerd}

%%%%%%%%%%%%%%%%%%%%%%%%%%%%%%
\section{Graphs as First Class Citizens}

\subsection{Connect and Spawn operations}

%%%%%%%
\par
\jaccode{jac_simple_walker.jac}{Simple walker creating and connected nodes}
\par
\jacdotnw{jac_simple_walker}{.3}{Graph in memory for JC~\ref{jac:jac_simple_walker.jac}}


%%%%%%%
\par
\jaccode{jac_named_node_edge.jac}{Creating named node types}
\par
\jacdotnw{jac_named_node_edge}{.3}{Graph in memory for JC~\ref{jac:jac_named_node_edge.jac}}

%%%%%%%
\par
\jaccode{jac_spawn_connect.jac}{Connecting nodes within spawn statement}
\par
\jacdotnw{jac_spawn_connect}{.3}{Graph in memory for JC~\ref{jac:jac_spawn_connect.jac}}

%%%%%%%
\par
\jaccode{jac_connect_chain.jac}{Chaining node connections using the connect operator}
\par
\jacdotnw{jac_connect_chain}{.3}{Graph in memory for JC~\ref{jac:jac_connect_chain.jac}}

Another incredibly useful notion to consider about connect operations is that they can be chained.
The same graph shown in Figure~\ref{dot:jac_connect_chain} can be achieved with the chained usage of the connect operation in line 8 of JC~\ref{jac:jac_connect_chain.jac}. Here nodes are chained in an intuitive left-to-right manor. Relatively sophisticated graph structures can be rapidly expressed using chained connect operations.

%%%%%%%%%%%%%%%%%%%%%%%%%%%%%%
\subsection{Static Graph Creation}

\subsubsection{Static Spawn Graphs}
%%%%%%%
\par
\jaccode{jac_static_spawn.jac}{A Spawn style static graph}

%%%%%%%
\par
\jaccode{jac_static_spawn_dot.jac}{Associated DOT style static graph}


\subsubsection{Static DOT Graphs}
%%%%%%%
\par
\jaccode{jac_static_dot.jac}{A DOT style static graph}
\par
\shellout{jac_static_dot.jac.output}
%%%%%%%
\par
\jaccode{jac_static_dot2.jac}{Another DOT style static graph}
\par
\shellout{jac_static_dot2.jac.output}




%%%%%%%%%%%%%%%%%%%%%%%%%%%%%%
\section{Walkers as the second First Class Citizens}

%%%%%%%
\par
\jaccode{jac_walker_spawn_walker.jac}{Walkers spawning other walkers}
\par
\shellout{jac_walker_spawn_walker.jac.output}
\par
\jacdotnw{jac_walker_spawn_walker}{.3}{Graph in memory for JC~\ref{jac:jac_walker_spawn_walker.jac}}

%%%%%%%
\par
\jaccode{jac_walker_returns.jac}{Getting returned values from spawned walkers}
\par
\shellout{jac_walker_returns.jac.output}
\par
\jacdotnw{jac_walker_returns}{.3}{Graph in memory for JC~\ref{jac:jac_walker_returns.jac}}
\par
\jaccode{jac_walker_returns_alt.jac}{Increasing elegance by remembering spawns are expressions}

%%%%%%%
Walkers are entry points to all valid jac programs
\par


%%%%%%%%%%%%%%%%%%%%%%%%%%%%%%
\section{Architypes}

\subsection{Context on Nodes and Edges}

%%%%%%%
\par
\jaccode{jac_contexts.jac}{Binding member contexts to nodes and edges
}
\par
\shellout{jac_contexts.jac.output}
\par
\jacdotnw{jac_contexts}{.3}{Graph in memory for JC~\ref{jac:jac_contexts.jac}}

%%%%%%%
\par
\jaccode{jac_contexts_alt.jac}{Binding contexts with less code
}
\par
\shellout{jac_contexts_alt.jac.output}

\subsection{Copy Assignment Operator}
%%%%%%%
\par
\jaccode{jac_copy_assign.jac}{Copy assigning from node to node
}
\par
\shellout{jac_copy_assign.jac.output}
\par
\jacdotnw{jac_copy_assign}{.5}{Graph in memory for JC~\ref{jac:jac_copy_assign.jac}}

\subsection{Plucking Values from Node and Edge Sets}

Another very handy dandy feature when interacting with collections of nodes and edges is to quickly extract a list of all the values for a given \lstinline{has} variable across the collection of nodes or edges. Lets look at an example.
\par
\jaccode{jac_arch_plucking.jac}{Plucking values out of nodes and edges}
\par
\shellout{jac_arch_plucking.jac.output}
\par
As shown in JC~\ref{jac:jac_arch_plucking.jac} we are referencing the \lstinline{has} variable of the architypes for the collection of \lstinline{simple} nodes and \lstinline{conn} edges on lines 8 and 9 respectively. As can be seen in the output, these references evaluate to a list of the values for the corresponding variables. Keep in mind this can work with a mixture of nodes and edges in a collection given they share a given \lstinline{has} variable name.

\subsection{Referencing and Dereferencing Nodes and Edges}

Nodes and edges can be referenced and dereferenced.
These operations are synonymous with they way references work in many languages and borrows the syntax of pointers in C/C++.
In particular, the \lstinline{&} is used to get the reference of an object and \lstinline{*} is used to dereference object.
However, in contrast to C/C++, instead of the references representing memory location in word format, references in Jac uses a unique identifier (in UUID format) for the object.
\par
\jaccode{jac_arch_ref_deref.jac}{Rereferences and dereferences in Jac}
\par
\shellout{jac_arch_ref_deref.jac.output}
\par
JC~\ref{jac:jac_arch_ref_deref.jac} shows an example of the behavior of references and dereferences in Jac.
Note that once dereferenced \lstinline{var} is simply a UUID formatted string with the unique identifier of the object itself.
This UUID is equivalent to the \lstinline{jid} in the object \lstinline{.info}.
These referencing and dereferencing operations are quite useful for input and output of node locations to a client side, etc.
\begin{nerd}
    Important Note: The internal representation of an instance of an architype is a string composed of any UUID that starts with \lstinline{"jac:uuid:"}. This may change in the future but, if you were to manually assign such a string to a variable in a Jac program, the program will treat this variable like an object.
\end{nerd}

%%%%%%%%%%%%%%%%%%%%%%%%%%%%%%
\section{Actions and Abilities}

\subsection{Actions}
%%%%%%%
\par
\jacdotnw{jac_action}{.3}{Graph in memory for JC~\ref{jac:jac_action.jac} and~\ref{jac:jac_node_action.jac}}
\par
\jaccode{jac_action.jac}{Basic action in walker
}
\par
\shellout{jac_action.jac.output}
\par
\jaccode{jac_node_action.jac}{Basic action in node
}

\subsection{Fused Interactions Between Nodes and Actions}
%%%%%%%
\par
\jaccode{jac_preset_action.jac}{Basic action with presets and event triggers
}
\par
\shellout{jac_preset_action.jac.output}
\par
\jaccode{jac_node_act_note.jac}{Basic action with presets and event triggers
}
\par [Only nodes can have with entry/exit`' and presets]
\par [can leave output (push returns) in node and walker]

\par
\shellout{jac_preset_action.jac.output}

\subsection{Abilities}
%%%%%%%
\par
\jaccode{jac_local_wlk_action.jac}{Actions and Abilities in Walkers
}
%%%%%%%
\par
\jaccode{jac_local_action.jac}{Abilities in nodes}


\subsection{here and visitor, the `this' references of Jac}
Observe the usage of \texttt{here} and \texttt{visitor} in the \texttt{person} node architype in JC~\ref{jac:jac_local_action.jac}.
These are synonymous to the this reference present in many other languages except \texttt{here} point to the current node scope relevant to the execution point in the program and \texttt{visitor} points to the relevant walker scope relevant to that given point of execution. These references provide full access to all \texttt{has} variables and builtin attributes and operations of the referenced object instance.
\par
Do note that in the context of the \texttt{person} node abilities in JC~\ref{jac:jac_local_action.jac} a here reference to say \texttt{here.name = "joe";} would be equivalent to simply \texttt{name = "joe";} however to capture the \texttt{here.context} (or info/details/etc) the \texttt{here} reference becomes quite useful. The similar relationship applies to using \texttt{visitor} in walker abilities.

%%%%%%%%%%%%%%%%%%%%%%%%%%%%%%
\section{Inheritance}
