\subsection{APIs for actions}

\par
This set action APIs enable the manual management of Jaseci actions and action
libraries/sets. Action libraries can be loaded locally into the running instance of
the python program, or as a remote container linked action library. In this mode,
action libraries operate as micro-services. Jaseci will be able to dynamically
and automatically make this decision for the user based on online monitoring and
performance profiling.

\subsubsection{\lstinline[basicstyle=\Large\ttfamily]$actions load local$}

\apispec{cli: actions load local | api: actions\_load\_local | auth: admin}{file: str (*req)}
{This API will dynamically load a module based on a python file. The module
is loaded directly into the running Jaseci python instance. This API also
makes an attempt to auto detect and hot load any python package dependencies
the file may reference via python's relative imports. This file is assumed to
have the necessary annotations and decorations required by Jaseci to recognize
its actions.\vspace{4mm}\par
\argspec{Parameters}{
\texttt{file} -- The python file with full to load actions from.
(i.e., ~/local/myact.py)\vspace{1.5mm}\par
}}
\subsubsection{\lstinline[basicstyle=\Large\ttfamily]$actions load remote$}

\apispec{cli: actions load remote | api: actions\_load\_remote | auth: admin}{url: str (*req)}
{This API will dynamically load a set of actions that are present on a remote
server/micro-service. This server must be configured to interact with Jaseci
properly. This is easily achieved using the same decorators used for local
action libraries. Remote actions allow for higher flexibility in the languages
supported for action libraries. If an  library writer would like to use another
language, the main hook REST api simply needs to be implemented. Please
refer to documentation on creating action libraries for more details.\vspace{4mm}\par
\argspec{Parameters}{
\texttt{url} -- The url of the API server supporting Jaseci actions.\vspace{1.5mm}\par
}}
\subsubsection{\lstinline[basicstyle=\Large\ttfamily]$actions load module$}

\apispec{cli: actions load module | api: actions\_load\_module | auth: admin}{mod: str (*req)}
{This API will dynamically load a module using python's module import format.
This is particularly useful for pip installed action libraries as the developer
can directly reference the module using the same format as a regular python
import. As with load local, the module will be loaded directly into the running
Jaseci python instance.\vspace{4mm}\par
\argspec{Parameters}{
\texttt{mod} -- The import style module to load actions from.
(i.e., jaseci\_kit.bi\_enc)\vspace{1.5mm}\par
}}
\subsubsection{\lstinline[basicstyle=\Large\ttfamily]$actions list$}

\apispec{cli: actions list | api: actions\_list | auth: admin}{name: str ()}
{This API is used to list the loaded actions active in Jaseci. These actions
include all types of loaded actions whether it be local modules or remote
containers. A particular set of actions can be viewed using the name parameter.\vspace{4mm}\par
\argspec{Parameters}{
\texttt{name} -- The name for a library for which to filter the view of shown
actions. If left blank all actions from all loaded sets will be shown.\vspace{1.5mm}\par
}}
\subsection{APIs for alias}

\par
The alias set of APIs provide a set of `alias' management functions for
creating and managing aliases for long strings such as UUIDs. If an alias'
name is used as a parameter value in any API call, that parameter will see
the alias' value instead. Given that references to all sentinels, walkers,
nodes, etc. utilize UUIDs, it becomes quite useful to create pneumonic
names for them. Also, when registering   sentinels, walkers, architype
handy aliases are automatically generated. These generated aliases can
then be managed using the alias APIs. Keep in mind that whenever an alias
is created, all parameter values submitted to any API with the alias name
will be replaced internally by its value. If you get in a bind, simply use
the clear or delete alias APIs.

\subsubsection{\lstinline[basicstyle=\Large\ttfamily]$alias register$}

\apispec{cli: alias register | api: alias\_register | auth: user}{name: str (*req), value: str (*req)}
{Either create new alias string to string mappings or replace
an existing mappings of a given alias name. Once registered the
alias mapping is instantly active.\vspace{4mm}\par
\argspec{Parameters}{
\texttt{name} -- The name for the alias created by caller.\vspace{1.5mm}\par

\texttt{value} -- The value for that name to map to (i.e., UUID)\vspace{1.5mm}\par
}\vspace{4mm}\par
\argspec{Returns}{Fields include
'response': Message of mapping that was created}}
\subsubsection{\lstinline[basicstyle=\Large\ttfamily]$alias list$}

\apispec{cli: alias list | api: alias\_list | auth: user}{n/a}
{Returns dictionary object of name to value mappings currently active.
This API is quite useful to track not only the aliases the caller
creates, but also the aliases automatically created as various Jaseci
objects (walkers, architypes, sentinels, etc.) are created, changed,
or destroyed.\vspace{4mm}\par
\argspec{Returns}{Dictionary of active mappings
'name': 'value'
...}}
\subsubsection{\lstinline[basicstyle=\Large\ttfamily]$alias delete$}

\apispec{cli: alias delete | api: alias\_delete | auth: user}{name: str (*req)}
{Removes a specific alias by its name. Only the alias is removed no
actual objects are affected. Future uses of this name will not be
mapped.\vspace{4mm}\par
\argspec{Parameters}{
\texttt{name} -- The name for the alias to be removed from caller.\vspace{1.5mm}\par
}\vspace{4mm}\par
\argspec{Returns}{Fields include
'response': Message of success/failure to remove alias
'success': True/False based on delete actually happening}}
\subsubsection{\lstinline[basicstyle=\Large\ttfamily]$alias clear$}

\apispec{cli: alias clear | api: alias\_clear | auth: user}{n/a}
{Removes a all aliases. No actual objects are affected. Aliases will
continue to be automatically generated when creating other Jaseci
objects.\vspace{4mm}\par
\argspec{Returns}{Fields include
'response': Message of number of alias removed
'removed': Number of aliases removed}}
\subsection{APIs for architype}

\par
The architype set of APIs allow for the addition and removing of
architypes. Given a Jac implementation of an architype these APIs are
designed for creating, compiling, and managing architypes that can be
used by Jaseci. There are two ways to add an architype to Jaseci, either
through the management of sentinels using the sentinel API, or by
registering independent architypes with these architype APIs. These
APIs are also used for inspecting and managing existing arichtypes that
a Jaseci instance is aware of.

\subsubsection{\lstinline[basicstyle=\Large\ttfamily]$architype register$}

\apispec{cli: architype register | api: architype\_register | auth: user}{code: str (*req), encoded: bool (False), snt: sentinel (None)}
{This register API allows for the creation or replacement/update of
an architype that can then be used by walkers in their interactions
of graphs. The code argument takes Jac source code for the single
architype. To load multiple architypes and walkers at the same time,
use sentinel register API.\vspace{4mm}\par
\argspec{Parameters}{
\texttt{code} -- The text (or filename) for an architypes Jac code\vspace{1.5mm}\par

\texttt{encoded} -- True/False flag as to whether code is encode
in base64\vspace{1.5mm}\par

\texttt{snt} -- The UUID of the sentinel to be the owner of this
architype\vspace{1.5mm}\par
}\vspace{4mm}\par
\argspec{Returns}{Fields include
'architype': Architype object if created otherwise null
'success': True/False whether register was successful
'errors': List of errors if register failed
'response': Message on outcome of register call}}
\subsubsection{\lstinline[basicstyle=\Large\ttfamily]$architype get$}

\apispec{cli: architype get | api: architype\_get | auth: user}{arch: architype (*req), mode: str (default), detailed: bool (False)}
{No documentation yet.\vspace{4mm}\par
\argspec{Parameters}{
\texttt{arch} -- The architype being accessed\vspace{1.5mm}\par

\texttt{mode} -- Valid modes: {default, code, ir, }\vspace{1.5mm}\par

\texttt{detailed} -- Flag to give summary or complete set of fields\vspace{1.5mm}\par
}\vspace{4mm}\par
\argspec{Returns}{Fields include (depends on mode)
'code': Formal source code for architype
'ir': Intermediate representation of architype
'architype': Architype object print}}
\subsubsection{\lstinline[basicstyle=\Large\ttfamily]$architype set$}

\apispec{cli: architype set | api: architype\_set | auth: user}{arch: architype (*req), code: str (*req), mode: str (default)}
{No documentation yet.\vspace{4mm}\par
\argspec{Parameters}{
\texttt{arch} -- The architype being set\vspace{1.5mm}\par

\texttt{code} -- The text (or filename) for an architypes Jac code/ir\vspace{1.5mm}\par

\texttt{mode} -- Valid modes: {default, code, ir, }\vspace{1.5mm}\par
}\vspace{4mm}\par
\argspec{Returns}{Fields include (depends on mode)
'success': True/False whether set was successful
'errors': List of errors if set failed
'response': Message on outcome of set call}}
\subsubsection{\lstinline[basicstyle=\Large\ttfamily]$architype list$}

\apispec{cli: architype list | api: architype\_list | auth: user}{snt: sentinel (None), detailed: bool (False)}
{No documentation yet.\vspace{4mm}\par
\argspec{Parameters}{
\texttt{snt} -- The sentinel for which to list its architypes\vspace{1.5mm}\par

\texttt{detailed} -- Flag to give summary or complete set of fields\vspace{1.5mm}\par
}\vspace{4mm}\par
\argspec{Returns}{List of architype objects}}
\subsubsection{\lstinline[basicstyle=\Large\ttfamily]$architype delete$}

\apispec{cli: architype delete | api: architype\_delete | auth: user}{arch: architype (*req), snt: sentinel (None)}
{No documentation yet.\vspace{4mm}\par
\argspec{Parameters}{
\texttt{arch} -- The architype being set\vspace{1.5mm}\par

\texttt{snt} -- The sentinel for which to list its architypes\vspace{1.5mm}\par
}\vspace{4mm}\par
\argspec{Returns}{Fields include (depends on mode)
'success': True/False whether command was successful
'response': Message on outcome of command}}
\subsection{APIs for config}

Abstracted since there are no valid configs in core atm, see jaseci\_serv
to see how used.

\subsubsection{\lstinline[basicstyle=\Large\ttfamily]$config get$}

\apispec{cli: config get | api: config\_get | auth: admin}{name: str (*req), do_check: bool (True)}
{No documentation yet.}
\subsubsection{\lstinline[basicstyle=\Large\ttfamily]$config set$}

\apispec{cli: config set | api: config\_set | auth: admin}{name: str (*req), value: str (*req), do_check: bool (True)}
{No documentation yet.}
\subsubsection{\lstinline[basicstyle=\Large\ttfamily]$config list$}

\apispec{cli: config list | api: config\_list | auth: admin}{n/a}
{No documentation yet.}
\subsubsection{\lstinline[basicstyle=\Large\ttfamily]$config index$}

\apispec{cli: config index | api: config\_index | auth: admin}{n/a}
{No documentation yet.}
\subsubsection{\lstinline[basicstyle=\Large\ttfamily]$config exists$}

\apispec{cli: config exists | api: config\_exists | auth: admin}{name: str (*req)}
{No documentation yet.}
\subsubsection{\lstinline[basicstyle=\Large\ttfamily]$config delete$}

\apispec{cli: config delete | api: config\_delete | auth: admin}{name: str (*req), do_check: bool (True)}
{No documentation yet.}
\subsection{APIs for global}

No documentation yet.

\subsubsection{\lstinline[basicstyle=\Large\ttfamily]$global set$}

\apispec{cli: global set | api: global\_set | auth: admin}{name: str (*req), value: str (*req)}
{No documentation yet.}
\subsubsection{\lstinline[basicstyle=\Large\ttfamily]$global delete$}

\apispec{cli: global delete | api: global\_delete | auth: admin}{name: str (*req)}
{No documentation yet.}
\subsubsection{\lstinline[basicstyle=\Large\ttfamily]$global sentinel set$}

\apispec{cli: global sentinel set | api: global\_sentinel\_set | auth: admin}{snt: sentinel (None)}
{No documentation yet.}
\subsubsection{\lstinline[basicstyle=\Large\ttfamily]$global sentinel unset$}

\apispec{cli: global sentinel unset | api: global\_sentinel\_unset | auth: admin}{n/a}
{No documentation yet.}
\subsection{APIs for graph}

No documentation yet.

\subsubsection{\lstinline[basicstyle=\Large\ttfamily]$graph create$}

\apispec{cli: graph create | api: graph\_create | auth: user}{set_active: bool (True)}
{No documentation yet.}
\subsubsection{\lstinline[basicstyle=\Large\ttfamily]$graph get$}

\apispec{cli: graph get | api: graph\_get | auth: user}{gph: graph (None), mode: str (default), detailed: bool (False)}
{Valid modes: {default, dot, }}
\subsubsection{\lstinline[basicstyle=\Large\ttfamily]$graph list$}

\apispec{cli: graph list | api: graph\_list | auth: user}{detailed: bool (False)}
{No documentation yet.}
\subsubsection{\lstinline[basicstyle=\Large\ttfamily]$graph active set$}

\apispec{cli: graph active set | api: graph\_active\_set | auth: user}{gph: graph (*req)}
{No documentation yet.}
\subsubsection{\lstinline[basicstyle=\Large\ttfamily]$graph active unset$}

\apispec{cli: graph active unset | api: graph\_active\_unset | auth: user}{n/a}
{No documentation yet.}
\subsubsection{\lstinline[basicstyle=\Large\ttfamily]$graph active get$}

\apispec{cli: graph active get | api: graph\_active\_get | auth: user}{detailed: bool (False)}
{No documentation yet.}
\subsubsection{\lstinline[basicstyle=\Large\ttfamily]$graph delete$}

\apispec{cli: graph delete | api: graph\_delete | auth: user}{gph: graph (*req)}
{No documentation yet.}
\subsubsection{\lstinline[basicstyle=\Large\ttfamily]$graph node get$}

\apispec{cli: graph node get | api: graph\_node\_get | auth: user}{nd: node (*req), keys: list ([])}
{No documentation yet.}
\subsubsection{\lstinline[basicstyle=\Large\ttfamily]$graph node set$}

\apispec{cli: graph node set | api: graph\_node\_set | auth: user}{nd: node (*req), ctx: dict (*req), snt: sentinel (None)}
{No documentation yet.}
\subsubsection{\lstinline[basicstyle=\Large\ttfamily]$graph walk$}

\apispec{cli: graph walk (cli only)}{nd: node (None)}
{No documentation yet.}
\subsection{APIs for jac}

No documentation yet.

\subsubsection{\lstinline[basicstyle=\Large\ttfamily]$jac build$}

\apispec{cli: jac build (cli only)}{file: str (*req), out: str ()}
{No documentation yet.}
\subsubsection{\lstinline[basicstyle=\Large\ttfamily]$jac test$}

\apispec{cli: jac test (cli only)}{file: str (*req), detailed: bool (False)}
{and .jir executables}
\subsubsection{\lstinline[basicstyle=\Large\ttfamily]$jac run$}

\apispec{cli: jac run (cli only)}{file: str (*req), walk: str (init), ctx: dict (\{\}), profiling: bool (False)}
{and .jir executables}
\subsubsection{\lstinline[basicstyle=\Large\ttfamily]$jac dot$}

\apispec{cli: jac dot (cli only)}{file: str (*req), walk: str (init), ctx: dict (\{\}), detailed: bool (False)}
{files and .jir executables}
\subsection{APIs for logger}

No documentation yet.

\subsubsection{\lstinline[basicstyle=\Large\ttfamily]$logger http connect$}

\apispec{cli: logger http connect | api: logger\_http\_connect | auth: admin}{host: str (*req), port: int (*req), url: str (*req), log: str (all)}
{Valid log params: {sys, app, all }}
\subsubsection{\lstinline[basicstyle=\Large\ttfamily]$logger http clear$}

\apispec{cli: logger http clear | api: logger\_http\_clear | auth: admin}{log: str (all)}
{Valid log params: {sys, app, all }}
\subsubsection{\lstinline[basicstyle=\Large\ttfamily]$logger list$}

\apispec{cli: logger list | api: logger\_list | auth: admin}{n/a}
{No documentation yet.}
\subsection{APIs for master}

\par
These APIs

\subsubsection{\lstinline[basicstyle=\Large\ttfamily]$master create$}

\apispec{cli: master create | api: master\_create | auth: user}{name: str (*req), global_init: str (), global_init_ctx: dict (\{\}), other_fields: dict (\{\})}
{other fields used for additional feilds for overloaded interfaces
(i.e., Dango interface)}
\subsubsection{\lstinline[basicstyle=\Large\ttfamily]$master get$}

\apispec{cli: master get | api: master\_get | auth: user}{name: str (*req), mode: str (default), detailed: bool (False)}
{Valid modes: {default, }}
\subsubsection{\lstinline[basicstyle=\Large\ttfamily]$master list$}

\apispec{cli: master list | api: master\_list | auth: user}{detailed: bool (False)}
{No documentation yet.}
\subsubsection{\lstinline[basicstyle=\Large\ttfamily]$master active set$}

\apispec{cli: master active set | api: master\_active\_set | auth: user}{name: str (*req)}
{NOTE: Specail handler included in general interface to api}
\subsubsection{\lstinline[basicstyle=\Large\ttfamily]$master active unset$}

\apispec{cli: master active unset | api: master\_active\_unset | auth: user}{n/a}
{No documentation yet.}
\subsubsection{\lstinline[basicstyle=\Large\ttfamily]$master active get$}

\apispec{cli: master active get | api: master\_active\_get | auth: user}{detailed: bool (False)}
{No documentation yet.}
\subsubsection{\lstinline[basicstyle=\Large\ttfamily]$master self$}

\apispec{cli: master self | api: master\_self | auth: user}{detailed: bool (False)}
{No documentation yet.}
\subsubsection{\lstinline[basicstyle=\Large\ttfamily]$master delete$}

\apispec{cli: master delete | api: master\_delete | auth: user}{name: str (*req)}
{No documentation yet.}
\subsection{APIs for object}

\par
...

\subsubsection{\lstinline[basicstyle=\Large\ttfamily]$global get$}

\apispec{cli: global get | api: global\_get | auth: user}{name: str (*req)}
{No documentation yet.}
\subsubsection{\lstinline[basicstyle=\Large\ttfamily]$object get$}

\apispec{cli: object get | api: object\_get | auth: user}{obj: element (*req), depth: int (0), detailed: bool (False)}
{No documentation yet.}
\subsubsection{\lstinline[basicstyle=\Large\ttfamily]$object perms get$}

\apispec{cli: object perms get | api: object\_perms\_get | auth: user}{obj: element (*req)}
{No documentation yet.}
\subsubsection{\lstinline[basicstyle=\Large\ttfamily]$object perms set$}

\apispec{cli: object perms set | api: object\_perms\_set | auth: user}{obj: element (*req), mode: str (*req)}
{No documentation yet.}
\subsubsection{\lstinline[basicstyle=\Large\ttfamily]$object perms default$}

\apispec{cli: object perms default | api: object\_perms\_default | auth: user}{mode: str (*req)}
{No documentation yet.}
\subsubsection{\lstinline[basicstyle=\Large\ttfamily]$object perms grant$}

\apispec{cli: object perms grant | api: object\_perms\_grant | auth: user}{obj: element (*req), mast: element (*req), read_only: bool (False)}
{No documentation yet.}
\subsubsection{\lstinline[basicstyle=\Large\ttfamily]$object perms revoke$}

\apispec{cli: object perms revoke | api: object\_perms\_revoke | auth: user}{obj: element (*req), mast: element (*req)}
{No documentation yet.}
\subsection{APIs for sentinel}

\par
A sentinel is a unit in Jaseci that represents the organization and management of
a collection of architypes and walkers. In a sense, you can think of a sentinel
as a complete Jac implementation of a program or API application. Though its the
case that many sentinels can be interchangeably across any set of graphs, most
use cases will typically be a single sentinel shared by all users and managed by an
admin(s), or each users maintaining a single sentinel customized for their
individual needs. Many novel usage models are possible, but I'd point the beginner
to the model most analogous to typical server side software development to start
with. This model would be to have a single admin account responsible for updating
a single sentinel that all users would share for their individual graphs. This
model is achieved through using \texttt{sentinel\_register},
\texttt{sentinel\_active\_global}, and \texttt{global\_sentinel\_set}.

\subsubsection{\lstinline[basicstyle=\Large\ttfamily]$sentinel register$}

\apispec{cli: sentinel register | api: sentinel\_register | auth: user}{name: str (default), code: str (), code_dir: str (./), mode: str (default), encoded: bool (False), auto_run: str (init), auto_run_ctx: dict (\{\}), auto_create_graph: bool (True), set_active: bool (True)}
{Auto run is the walker to execute on register (assumes active graph
is selected)}
\subsubsection{\lstinline[basicstyle=\Large\ttfamily]$sentinel pull$}

\apispec{cli: sentinel pull | api: sentinel\_pull | auth: user}{set_active: bool (True), on_demand: bool (True)}
{No documentation yet.}
\subsubsection{\lstinline[basicstyle=\Large\ttfamily]$sentinel get$}

\apispec{cli: sentinel get | api: sentinel\_get | auth: user}{snt: sentinel (None), mode: str (default), detailed: bool (False)}
{Valid modes: {default, code, ir, }}
\subsubsection{\lstinline[basicstyle=\Large\ttfamily]$sentinel set$}

\apispec{cli: sentinel set | api: sentinel\_set | auth: user}{code: str (*req), code_dir: str (./), encoded: bool (False), snt: sentinel (None), mode: str (default)}
{Valid modes: {code, ir, }}
\subsubsection{\lstinline[basicstyle=\Large\ttfamily]$sentinel list$}

\apispec{cli: sentinel list | api: sentinel\_list | auth: user}{detailed: bool (False)}
{No documentation yet.}
\subsubsection{\lstinline[basicstyle=\Large\ttfamily]$sentinel test$}

\apispec{cli: sentinel test | api: sentinel\_test | auth: user}{snt: sentinel (None), detailed: bool (False)}
{No documentation yet.}
\subsubsection{\lstinline[basicstyle=\Large\ttfamily]$sentinel active set$}

\apispec{cli: sentinel active set | api: sentinel\_active\_set | auth: user}{snt: sentinel (*req)}
{No documentation yet.}
\subsubsection{\lstinline[basicstyle=\Large\ttfamily]$sentinel active unset$}

\apispec{cli: sentinel active unset | api: sentinel\_active\_unset | auth: user}{n/a}
{No documentation yet.}
\subsubsection{\lstinline[basicstyle=\Large\ttfamily]$sentinel active global$}

\apispec{cli: sentinel active global | api: sentinel\_active\_global | auth: user}{auto_run: str (), auto_run_ctx: dict (\{\}), auto_create_graph: bool (False), detailed: bool (False)}
{Exclusive OR with pull strategy}
\subsubsection{\lstinline[basicstyle=\Large\ttfamily]$sentinel active get$}

\apispec{cli: sentinel active get | api: sentinel\_active\_get | auth: user}{detailed: bool (False)}
{No documentation yet.}
\subsubsection{\lstinline[basicstyle=\Large\ttfamily]$sentinel delete$}

\apispec{cli: sentinel delete | api: sentinel\_delete | auth: user}{snt: sentinel (*req)}
{No documentation yet.}
\subsection{APIs for stripe}

Set of APIs to expose jaseci stripe management

\subsubsection{\lstinline[basicstyle=\Large\ttfamily]$stripe product create$}

\apispec{cli: stripe product create | api: stripe\_product\_create | auth: admin}{name: str (VIP Plan), description: str (Plan description)}
{No documentation yet.}
\subsubsection{\lstinline[basicstyle=\Large\ttfamily]$stripe product price set$}

\apispec{cli: stripe product price set | api: stripe\_product\_price\_set | auth: admin}{productId: str (*req), amount: float (50), interval: str (month)}
{No documentation yet.}
\subsubsection{\lstinline[basicstyle=\Large\ttfamily]$stripe product list$}

\apispec{cli: stripe product list | api: stripe\_product\_list | auth: admin}{detalied: bool (True)}
{No documentation yet.}
\subsubsection{\lstinline[basicstyle=\Large\ttfamily]$stripe customer create$}

\apispec{cli: stripe customer create | api: stripe\_customer\_create | auth: admin}{paymentId: str (*req), name: str (cristopher evangelista), email: str (imurbatman12@gmail.com), description: str (Myca subscriber)}
{No documentation yet.}
\subsubsection{\lstinline[basicstyle=\Large\ttfamily]$stripe customer get$}

\apispec{cli: stripe customer get | api: stripe\_customer\_get | auth: admin}{customerId: str (*req)}
{No documentation yet.}
\subsubsection{\lstinline[basicstyle=\Large\ttfamily]$stripe customer payment add$}

\apispec{cli: stripe customer payment add | api: stripe\_customer\_payment\_add | auth: admin}{paymentMethodId: str (*req), customerId: str (*req)}
{No documentation yet.}
\subsubsection{\lstinline[basicstyle=\Large\ttfamily]$stripe customer payment delete$}

\apispec{cli: stripe customer payment delete | api: stripe\_customer\_payment\_delete | auth: admin}{paymentMethodId: str (*req)}
{No documentation yet.}
\subsubsection{\lstinline[basicstyle=\Large\ttfamily]$stripe customer payment get$}

\apispec{cli: stripe customer payment get | api: stripe\_customer\_payment\_get | auth: admin}{customerId: str (*req)}
{No documentation yet.}
\subsubsection{\lstinline[basicstyle=\Large\ttfamily]$stripe customer payment default$}

\apispec{cli: stripe customer payment default | api: stripe\_customer\_payment\_default | auth: admin}{customerId: str (*req), paymentMethodId: str (*req)}
{No documentation yet.}
\subsubsection{\lstinline[basicstyle=\Large\ttfamily]$stripe subscription create$}

\apispec{cli: stripe subscription create | api: stripe\_subscription\_create | auth: admin}{paymentId: str (*req), name: str (*req), email: str (*req), priceId: str (*req), customerId: str (*req)}
{TODO: name and email parameters not used!}
\subsubsection{\lstinline[basicstyle=\Large\ttfamily]$stripe subscription delete$}

\apispec{cli: stripe subscription delete | api: stripe\_subscription\_delete | auth: admin}{subscriptionId: str (*req)}
{No documentation yet.}
\subsubsection{\lstinline[basicstyle=\Large\ttfamily]$stripe subscription get$}

\apispec{cli: stripe subscription get | api: stripe\_subscription\_get | auth: admin}{customerId: str (*req)}
{No documentation yet.}
\subsubsection{\lstinline[basicstyle=\Large\ttfamily]$stripe invoices list$}

\apispec{cli: stripe invoices list | api: stripe\_invoices\_list | auth: admin}{customerId: str (*req), subscriptionId: str (*req), limit: int (10), lastItem: str ()}
{No documentation yet.}
\subsection{APIs for super}

No documentation yet.

\subsubsection{\lstinline[basicstyle=\Large\ttfamily]$master createsuper$}

\apispec{cli: master createsuper | api: master\_createsuper | auth: admin}{name: str (*req), global_init: str (), global_init_ctx: dict (\{\}), other_fields: dict (\{\})}
{other fields used for additional feilds for overloaded interfaces
(i.e., Dango interface)}
\subsubsection{\lstinline[basicstyle=\Large\ttfamily]$master allusers$}

\apispec{cli: master allusers | api: master\_allusers | auth: admin}{num: int (0), start_idx: int (0)}
{return and start idx specfies where to start
NOTE: Abstract interface to be overridden}
\subsubsection{\lstinline[basicstyle=\Large\ttfamily]$master become$}

\apispec{cli: master become | api: master\_become | auth: admin}{mast: master (*req)}
{No documentation yet.}
\subsubsection{\lstinline[basicstyle=\Large\ttfamily]$master unbecome$}

\apispec{cli: master unbecome | api: master\_unbecome | auth: admin}{n/a}
{No documentation yet.}
\subsection{APIs for user}

\par
These User APIs enable the creation and management of users on a Jaseci machine.
The creation of a user in this context is synonymous to the creation of a master
Jaseci object. These APIs are particularly useful when running a Jaseci server
or cluster in contrast to running JSCTL on the command line. Upon executing JSCTL
a dummy admin user (super\_master) is created and all state is dumped to a session
file, though any users created during a JSCTL session will indeed be created as
part of that session's state.

\subsubsection{\lstinline[basicstyle=\Large\ttfamily]$user create$}

\apispec{cli: user create | api: user\_create | auth: public}{name: str (*req), global_init: str (), global_init_ctx: dict (\{\}), other_fields: dict (\{\})}
{This API is used to create users and optionally set them up with a graph and
related initialization. In the context of
JSCTL, any name is sufficient and no additional information is required.
However, for Jaseci serving (whether it be the official Jaseci server, or a
custom overloaded server) additional fields are required and should be added
to the other fields parameter as per the specifics of the encapsulating server
requirements. In the case of the official Jaseci server, the name field must
be a valid email, and a password field must be passed through other fields.
A number of other optional parameters can also be passed through other feilds.
\vspace{3mm}\par
This single API call can also be used to fully set up and initialize a user
by leveraging the global init parameter. When set, this parameter attaches the
user to the global sentinel, creates a new graph for the user, sets it as the
active graph, then runs an initialization walker on the root node of this new
graph. The initialization walker is identified by the name assigned to
global init. The default empty string assigned to global init indicates this
global setup should not be run.\vspace{4mm}\par
\argspec{Parameters}{
\texttt{name} -- The user name to create. For Jaseci server this must be a valid
email address.\vspace{1.5mm}\par

\texttt{global\_init} -- The name of an initialization walker. When set the user is
linked to the global sentinel and the walker is run on a new active graph
created for the user.\vspace{1.5mm}\par

\texttt{global\_init\_ctx} -- Context to preload for the initialization walker\vspace{1.5mm}\par

\texttt{other\_fields} -- This parameter is used for additional fields required for
overloaded interfaces. This parameter is not used in JSCTL, but is used
by Jaseci server for the additional parameters of password, is\_activated,
and is\_superuser.\vspace{1.5mm}\par
}}
\subsection{APIs for walker}

\par
The walker set of APIs are used for execution and management of walkers. Walkers
are the primary entry points for running Jac programs. The
primary API used to run walkers is \textbf{walker\_run}. There are a number of
variations on this API that enable the invocation of walkers with various
semantics.

\subsubsection{\lstinline[basicstyle=\Large\ttfamily]$walker register$}

\apispec{cli: walker register | api: walker\_register | auth: user}{snt: sentinel (None), code: str (), encoded: bool (False)}
{Though the common case is to register entire sentinels, a user can also
register individual walkers one at a time. This API accepts code for a single
walker (i.e., \lstinline\{walker \{...\}\}).}
\subsubsection{\lstinline[basicstyle=\Large\ttfamily]$walker get$}

\apispec{cli: walker get | api: walker\_get | auth: user}{wlk: walker (*req), mode: str (default), detailed: bool (False)}
{Valid modes: {default, code, ir, keys, }}
\subsubsection{\lstinline[basicstyle=\Large\ttfamily]$walker set$}

\apispec{cli: walker set | api: walker\_set | auth: user}{wlk: walker (*req), code: str (*req), mode: str (default)}
{Valid modes: {code, ir, }}
\subsubsection{\lstinline[basicstyle=\Large\ttfamily]$walker list$}

\apispec{cli: walker list | api: walker\_list | auth: user}{snt: sentinel (None), detailed: bool (False)}
{No documentation yet.}
\subsubsection{\lstinline[basicstyle=\Large\ttfamily]$walker delete$}

\apispec{cli: walker delete | api: walker\_delete | auth: user}{wlk: walker (*req), snt: sentinel (None)}
{No documentation yet.}
\subsubsection{\lstinline[basicstyle=\Large\ttfamily]$walker spawn create$}

\apispec{cli: walker spawn create | api: walker\_spawn\_create | auth: user}{name: str (*req), snt: sentinel (None)}
{No documentation yet.}
\subsubsection{\lstinline[basicstyle=\Large\ttfamily]$walker spawn list$}

\apispec{cli: walker spawn list | api: walker\_spawn\_list | auth: user}{detailed: bool (False)}
{No documentation yet.}
\subsubsection{\lstinline[basicstyle=\Large\ttfamily]$walker spawn delete$}

\apispec{cli: walker spawn delete | api: walker\_spawn\_delete | auth: user}{name: str (*req)}
{No documentation yet.}
\subsubsection{\lstinline[basicstyle=\Large\ttfamily]$walker spawn clear$}

\apispec{cli: walker spawn clear | api: walker\_spawn\_clear | auth: user}{n/a}
{No documentation yet.}
\subsubsection{\lstinline[basicstyle=\Large\ttfamily]$walker yield list$}

\apispec{cli: walker yield list | api: walker\_yield\_list | auth: user}{detailed: bool (False)}
{No documentation yet.}
\subsubsection{\lstinline[basicstyle=\Large\ttfamily]$walker yield delete$}

\apispec{cli: walker yield delete | api: walker\_yield\_delete | auth: user}{name: str (*req)}
{No documentation yet.}
\subsubsection{\lstinline[basicstyle=\Large\ttfamily]$walker yield clear$}

\apispec{cli: walker yield clear | api: walker\_yield\_clear | auth: user}{n/a}
{No documentation yet.}
\subsubsection{\lstinline[basicstyle=\Large\ttfamily]$walker prime$}

\apispec{cli: walker prime | api: walker\_prime | auth: user}{wlk: walker (*req), nd: node (None), ctx: dict (\{\}), _req_ctx: dict (\{\})}
{No documentation yet.}
\subsubsection{\lstinline[basicstyle=\Large\ttfamily]$walker execute$}

\apispec{cli: walker execute | api: walker\_execute | auth: user}{wlk: walker (*req), prime: node (None), ctx: dict (\{\}), _req_ctx: dict (\{\}), profiling: bool (False)}
{No documentation yet.}
\subsubsection{\lstinline[basicstyle=\Large\ttfamily]$walker run$}

\apispec{cli: walker run | api: walker\_run | auth: user}{name: str (*req), nd: node (None), ctx: dict (\{\}), _req_ctx: dict (\{\}), snt: sentinel (None), profiling: bool (False), is_async: bool (False)}
{reports results, and cleans up walker instance.}
\subsubsection{\lstinline[basicstyle=\Large\ttfamily]$wapi$}

\apispec{cli: wapi | api: wapi | auth: user}{name: str (*req), nd: node (None), ctx: dict (\{\}), _req_ctx: dict (\{\}), snt: sentinel (None), profiling: bool (False)}
{No documentation yet.}
\subsubsection{\lstinline[basicstyle=\Large\ttfamily]$walker summon$}

\apispec{cli: walker summon | api: walker\_summon | auth: public}{key: str (*req), wlk: walker (*req), nd: node (*req), ctx: dict (\{\}), _req_ctx: dict (\{\}), global_sync: bool (True)}
{along with the walker id and node id}
\subsubsection{\lstinline[basicstyle=\Large\ttfamily]$walker callback$}

\apispec{cli: walker callback | api: walker\_callback | auth: public}{nd: node (*req), wlk: walker (*req), key: str (*req), ctx: dict (\{\}), _req_ctx: dict (\{\}), global_sync: bool (True)}
{along with the walker id and node id}
\subsubsection{\lstinline[basicstyle=\Large\ttfamily]$walker queue$}

\apispec{cli: walker queue | api: walker\_queue | auth: user}{task_id: str ()}
{No documentation yet.}
