\subsection{date}
\par
No documentation yet.
\apispec{date.quantize\_to\_year}{date: str (*req)}
{No documentation yet.}
\apispec{date.quantize\_to\_month}{date: str (*req)}
{No documentation yet.}
\apispec{date.quantize\_to\_week}{date: str (*req)}
{No documentation yet.}
\apispec{date.quantize\_to\_day}{date: str (*req)}
{No documentation yet.}
\apispec{date.date\_day\_diff}{start_date: str (*req), end_date: str (*req)}
{No documentation yet.}
\subsection{file}
\par
No documentation yet.
\apispec{file.load\_str}{fn: str (*req), max_chars: int}
{No documentation yet.}
\apispec{file.load\_json}{fn: str (*req)}
{No documentation yet.}
\apispec{file.dump\_str}{fn: str (*req), s: str (*req)}
{No documentation yet.}
\apispec{file.append\_str}{fn: str (*req), s: str (*req)}
{No documentation yet.}
\apispec{file.dump\_json}{fn: str (*req), obj: _empty (*req), indent: int}
{No documentation yet.}
\apispec{file.delete}{fn: str (*req)}
{No documentation yet.}
\subsection{net}
\par
This library of actions cover the standard operations that can be
run on graph elements (nodes and edges). A number of these actions
accept lists that are exclusively composed of instances of defined
architype node and/or edges. Keep in mind that a \lstinline{jac_set}
is simply a list that only contains such elements.
\apispec{net.max}{item_set: jac_set (*req)}
{This action will return the maximum element in a list of nodes
and/or edges. This action exclusively utilizes the anchor variable
of the node/edge arhcitype as the representative field for
performing the  comparison in ranking. This action does not support
arhcitypes lacking an anchor.\vspace{4mm}\par
\argspec{Parameters}{
\texttt{item\_set} -- A list of node and or edges to identify the
maximum element based on their respective anchor values\vspace{1.5mm}\par
}\vspace{4mm}\par
\argspec{Returns}{A node or edge object}}
\apispec{net.min}{item_set: jac_set (*req)}
{This action will return the minimum element in a list of nodes
and/or edges. This action exclusively utilizes the anchor variable
of the node/edge arhcitype as the representative field for
performing the  comparison in ranking. This action does not support
arhcitypes lacking an anchor.\vspace{4mm}\par
\argspec{Parameters}{
\texttt{item\_set} -- A list of node and or edges to identify the
minimum element based on their respective anchor values\vspace{1.5mm}\par
}\vspace{4mm}\par
\argspec{Returns}{A node or edge object}}
\apispec{net.pack}{item_set: jac_set (*req)}
{No documentation yet.}
\apispec{net.unpack}{item_set: jac_set (*req)}
{No documentation yet.}
\apispec{net.root}{n/a}
{This action returns the root node for the graph of a given user (master). A call
to this action is only valid if the user has an active graph set, otherwise it
return null. This is a handy way for any walker to get to the root node of a
graph.}
\subsection{rand}
\par
No documentation yet.
\apispec{rand.seed}{val: int (*req)}
{No documentation yet.}
\apispec{rand.integer}{start: int (*req), end: int (*req)}
{No documentation yet.}
\apispec{rand.choice}{lst: list (*req)}
{No documentation yet.}
\apispec{rand.sentence}{min_lenth: int, max_length: int, sep: str}
{No documentation yet.}
\apispec{rand.paragraph}{min_lenth: int, max_length: int, sep: str}
{No documentation yet.}
\apispec{rand.text}{min_lenth: int, max_length: int, sep: str}
{No documentation yet.}
\apispec{rand.word}{n/a}
{No documentation yet.}
\apispec{rand.time}{start_date: str (*req), end_date: str (*req)}
{No documentation yet.}
\subsection{request}
\par
No documentation yet.
\apispec{request.get}{url: str (*req), data: dict (*req), header: dict (*req)}
{Param 1 - url
Param 2 - data
Param 3 - header

Return - response object}
\apispec{request.post}{url: str (*req), data: dict (*req), header: dict (*req)}
{Param 1 - url
Param 2 - data
Param 3 - header

Return - response object}
\apispec{request.put}{url: str (*req), data: dict (*req), header: dict (*req)}
{Param 1 - url
Param 2 - data
Param 3 - header

Return - response object}
\apispec{request.delete}{url: str (*req), data: dict (*req), header: dict (*req)}
{Param 1 - url
Param 2 - data
Param 3 - header

Return - response object}
\apispec{request.head}{url: str (*req), data: dict (*req), header: dict (*req)}
{Param 1 - url
Param 2 - data
Param 3 - header

Return - response object}
\apispec{request.options}{url: str (*req), data: dict (*req), header: dict (*req)}
{Param 1 - url
Param 2 - data
Param 3 - header

Return - response object}
\apispec{request.multipart\_base64}{url: str (*req), files: list (*req), header: dict (*req)}
{Param 1 - url
Param 3 - header
Param 3 - file (Optional) used for single file
Param 4 - files (Optional) used for multiple files
Note - file and files can't be None at the same time

Return - response object}
\apispec{request.file\_download\_base64}{url: str (*req), header: dict (*req), encoding: str}
{No documentation yet.}
\subsection{std}
\par
No documentation yet.
\apispec{std.log}{args: _empty (*req)}
{No documentation yet.}
\apispec{std.out}{args: _empty (*req)}
{No documentation yet.}
\apispec{std.js\_input}{prompt: str}
{No documentation yet.}
\apispec{std.err}{args: _empty (*req)}
{No documentation yet.}
\apispec{std.sort\_by\_col}{lst: list (*req), col_num: int (*req), reverse: bool}
{Param 1 - list
Param 2 - col number
Param 3 - boolean as to whether things should be reversed

Return - Sorted list}
\apispec{std.time\_now}{n/a}
{No documentation yet.}
\apispec{std.set\_global}{name: str (*req), value: _empty (*req)}
{Param 1 - name
Param 2 - value (must be json serializable)}
\apispec{std.get\_global}{name: str (*req)}
{Param 1 - name}
\apispec{std.actload\_local}{filename: str (*req)}
{No documentation yet.}
\apispec{std.actload\_remote}{url: str (*req)}
{No documentation yet.}
\apispec{std.actload\_module}{module: str (*req)}
{No documentation yet.}
\apispec{std.destroy\_global}{name: str (*req)}
{No documentation yet.}
\apispec{std.set\_perms}{obj: element (*req), mode: str (*req)}
{Param 1 - target element
Param 2 - valid permission (public, private, read only)

Return - true/false whether successful}
\apispec{std.get\_perms}{obj: element (*req)}
{Param 1 - target element

Return - Sorted list}
\apispec{std.grant\_perms}{obj: element (*req), mast: element (*req), read_only: bool (*req)}
{Param 1 - target element
Param 2 - master to be granted permission
Param 3 - Boolean read only flag

Return - Sorted list}
\apispec{std.revoke\_perms}{obj: element (*req), mast: element (*req)}
{Param 1 - target element
Param 2 - master to be revoked permission

Return - Sorted list}
\apispec{std.get\_report}{n/a}
{No documentation yet.}
\subsection{vector}
\par
No documentation yet.
\apispec{vector.cosine\_sim}{vec_a: list (*req), vec_b: list (*req)}
{Param 1 - First vector
Param 2 - Second vector

Return - float between 0 and 1}
\apispec{vector.dot\_product}{vec_a: list (*req), vec_b: list (*req)}
{Param 1 - First vector
Param 2 - Second vector

Return - float between 0 and 1}
\apispec{vector.get\_centroid}{vec_list: list (*req)}
{Param 1 - List of vectors

Return - (centroid vector, cluster tightness)}
\apispec{vector.softmax}{vec_list: list (*req)}
{Param 1 - List of vectors

Return - (centroid vector, cluster tightness)}
\apispec{vector.sort\_by\_key}{data: dict (*req), reverse: _empty, key_pos: _empty}
{Param 1 - List of items
Param 2 - if Reverse
Param 3 (Optional) - Index of the key to be used for sorting
if param 1 is a list of tuples.

Deprecated}
