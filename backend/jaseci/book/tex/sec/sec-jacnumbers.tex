%%%%%%%%%%%%%%%%%%%%%%%%%%%%%%
\subsection*{Basic Arithmetic Operations}
The simplest math operations in Jac.
\begin{description}
    \item[Code] \texttt{}
          \begin{lstlisting}[caption={Basic arithmetic operations}]
walker init {
    a = 4 + 4;
    b = 4 * -5;
    c = 4 / 4;  # Evaluates to a floating point number
    d = 4 - 6;
    e = a + b + c + d;
    std.out(a, b, c, d, e);
}
    \end{lstlisting}
    \item[Output] \texttt{ }
          \begin{lstlisting}[language=shell]
8 -20 1.0 -2 -13.0
        \end{lstlisting}
    \item[Description] \texttt{}
\end{description}

\noindent Additionally, Jac supports power and modulo operations.
\begin{description}
    \begin{lstlisting}[caption={Additional arithmetic operations}]
walker init {
    a = 4 ^ 4; b = 9 % 5; std.out(a, b);
}
    \end{lstlisting}
    \item[Output] \texttt{ }
          \begin{lstlisting}[language=shell]
256 4
        \end{lstlisting}
    \item[Description] \texttt{}
\end{description}


%%%%%%%%%%%%%%%%%%%%%%%%%%%%%%
\subsection*{Comparison Operations}
\begin{description}
    \begin{lstlisting}[caption={Comparision operations}]
walker init {
    a = 5; b = 6;
    std.out(a == b,
            a != b,
            a < b,
            a > b,
            a <= b,
            a >= b,
            a == b-1);
}
    \end{lstlisting}
    \item[Output] \texttt{ }
          \begin{lstlisting}[language=shell]
false true true false true false true
        \end{lstlisting}
    \item[Description] \texttt{}
\end{description}


%%%%%%%%%%%%%%%%%%%%%%%%%%%%%%
\subsection*{Logical Operations}
\begin{description}
    \begin{lstlisting}[caption={Logical operations}]
walker init {
    a = true; b = false;
    std.out(a,
            !a,
            a && b,
            a || b,
            a and b,
            a or b,
            !a or b,
            !(a and b));
}
    \end{lstlisting}
    \item[Output] \texttt{ }
          \begin{lstlisting}[language=shell]
true false false true false true false true
        \end{lstlisting}
    \item[Description] \texttt{}
\end{description}


%%%%%%%%%%%%%%%%%%%%%%%%%%%%%%
\subsection*{Assignment Operations}
\begin{description}
    \begin{lstlisting}[caption={Assignment operations}]
walker init {
    a = 4 + 4; std.out(a);
    a += 4 + 4; std.out(a);
    a -= 4 * -5; std.out(a);
    a *= 4 / 4; std.out(a);
    a /= 4 - 6; std.out(a);

    # a := here; std.out(a);
    # Noting existence of copy assign, described later
}
    \end{lstlisting}
    \item[Output] \texttt{ }
          \begin{lstlisting}[language=shell]
8
16
36
36.0
-18.0
        \end{lstlisting}
    \item[Description] \texttt{}
\end{description}


%%%%%%%%%%%%%%%%%%%%%%%%%%%%%%
\subsection*{Foreshadowing Unique Graph Operations}
\begin{description}
    \begin{lstlisting}[caption={Preview of graph operators},
        label={code:moremath}]
edge back;

walker init {
    node_a = spawn node;
    here --> node_a;
    here <-[back]- node_a;

    node_b = spawn here <-> node;
    node_b --> node_a
}
    \end{lstlisting}
    \item[Output] \texttt{ }
    \item[Description] \texttt{}

          \begin{tikzpicture}[node distance = {1.0cm and 1.5cm}, v/.style = {draw, circle}]
              \graph[nodes={circle, draw}, grow right=2.25cm, branch down=1.75cm]{
              H -> A,
              A -> ["back"] H,
              H -- B,
              B -> A,
              };
          \end{tikzpicture}
\end{description}


%%%%%%%%%%%%%%%%%%%%%%%%%%%%%%
\subsection*{Precedence}

\begin{table}[h]
    \small
    \centering
    \begin{tabular}{l l l}
        \toprule
        \textbf{Rank} & \textbf{Symbol}          & \textbf{Description}                           \\
        \midrule
        1             & () [] . :: --> <-- spawn & Parenthetical/grouping, node/edge manipulation \\
        2             & \textasciicircum         & Exponent                                       \\
        3             & * / \%                   & Multiplication, division, modulo               \\
        4             & + -                      & Addition, subtraction                          \\
        5             & == != >= <= > <          & Comparison                                     \\
        6             & \&\& || and or           & Logical                                        \\
        7             & = += -= *= /= :=         & Assignment                                     \\
        \bottomrule
    \end{tabular}
    \caption{Precedence of operations in Jac}
    \label{tab:jacprecedence} % Unique label used for referencing the table in-text
    %\addcontentsline{toc}{table}{Table \ref{tab:jacprecedence}} % Uncomment to add the table to the table of contents
\end{table}